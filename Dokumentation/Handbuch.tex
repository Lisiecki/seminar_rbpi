\documentclass[12pt,a4paper]{scrreprt}
\usepackage[utf8]{inputenc}
\usepackage{amsmath}
\usepackage{amsfonts}
\usepackage{amssymb}

\title{Handbuch \\ Gemeinsame, verteilte Überwachung eines Gebietes auf Eindringlinge}
\author{Dennis Lisiecki, Torsten Kühl}

\begin{document}

\maketitle	%Titelblatt erstellen
\tableofcontents	%Inhaltsverzeichnis erstellen
\chapter{Vorbereitung}
\section{Diese Teile werden benötigt}
\begin{itemize}
\item Raspberry Pi B+
\item SD-Karte mit 8GB
\item Strom für den Raspi
\item Monitor für Ersteinrichtung
\item Tastatur und Maus
\item Pi NoIR-Kamera
\item PIR-Modul
\item Steckplatine
\item Ein Paket Kabel
\end{itemize}

\section{Gib mir einen Namen}
Im folgendem gehen wir davon aus, dass der Raspberry-Pi bzw. die SD-Karte bereits mit der aktuellsten Version vom Raspbian-OS ausgestattet ist. Zunächst ist es immer ratsam den Raspberry per Befehl \textit{sudo apt-get update} und \textit{sudo apt-get upgrade} auf den neuesten Stand zu bringen.


\end{document}