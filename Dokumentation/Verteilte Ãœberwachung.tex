\documentclass[12pt,a4paper]{scrreprt}
\usepackage[utf8]{inputenc}
\usepackage{amsmath}
\usepackage{amsfonts}
\usepackage{amssymb}

\title{Kooperative, verteilte Überwachung eines Gebiets auf Eindringlinge}
\author{Dennis Lisiecki, Torsten Kühl}

\begin{document}

\maketitle	%Titelblatt erstellen
\tableofcontents	%Inhaltsverzeichnis erstellen

\chapter{Einleitung}
\section{Anforderungen / Lastenheft}
Ziel ist es, mithilfe von 2 (oder mehr) Raspberry-Pis, die jeweils mit Kamera und Infrarot-Sensoren bestückt sind, ein Gebiet oder einen Raum auf unerwünschte Eindringlinge zu überprüfen. 
Dazu werden die Geräte an unterschiedlichen Positionen aufgestellt, welche jeweils den gleichen Bereich überwachen sollen. 
\\
\\
Die Geräte sollen mithilfe der Kamera und eines Schrittmotoren dazu in der Lage sein, einem anvisierten Ziel automatisch zu folgen, um dieses zu beobachten und es aufzuzeichnen. Um unerwünschte Aufzeichnungen zu vermeiden (z.B. durch sich plötzlich ändernde Lichtverhältnisse), soll die Kamera mit einem PIR-Modul kooperieren.
Wenn auf diese Art und Weise 2 oder mehr Geräte eine Bewegung feststellen, soll ein Alarm auslösen.



\section{Bestehende Lösungen}
%Hier kommt der Text

\chapter{Technische Realisierung}
\section{Benötigte Module}
%Hier kommt der Text

\section{Benutzte Open-Source Software}
%Hier kommt der Text

\chapter{Praktische Realisierung}
\section{Kommunikation der Kameras}
%Hier kommt der Text


\chapter{Erklärung der Algorithmen}
\section{Funktionsweise der Hard- und Software}
%Hier kommt der Text

\chapter{Realisierung als verteiltes System}
\section{Anforderungen}
%Hier kommt der Text

\section{Praktische Realisierung}
%Hier kommt der Text

\section{Vorteile}
%Hier kommt der Text


\end{document}