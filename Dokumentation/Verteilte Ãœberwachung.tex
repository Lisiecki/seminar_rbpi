\documentclass[journal]{IEEEtran}
\usepackage[utf8]{inputenc}
\usepackage[ngerman]{babel}

\usepackage{amsmath}
\usepackage{amsfonts}
\usepackage{amssymb}
\usepackage{todonotes}	%todonotes für die toto-annotationen im Text und in der Datei

%% BIBLATEX EINBINDEN 
\usepackage[backend=bibtex]{biblatex} 
\bibliography{refs}

\title{Kooperative, verteilte Überwachung eines Gebiets auf Eindringlinge}		%Beschriftung für Titelblatt
\author{Dennis Lisiecki, Torsten Kühl}								%Ebenso, nur kleiner

\begin{document}	%Hier beginnt das Dokument

\maketitle	%Titelblatt erstellen
%\tableofcontents	%Inhaltsverzeichnis erstellen....NICHT

%\chapter{Einleitung}



\section{Einleitung}
Eine Überwachungskamera dient im allgemeinem dem Zweck, einen bestimmten Bereich dauerhaft zu überwachen. Das Bild wird kann direkt auf einem Monitor wiedergegeben und parallel aufgezeichnet werden, in Supermärkten erfolgt die Ausstrahlung des Live-Bilds sogar zur Abschreckung oftmals direkt im Verkaufsraum. Zum Schutz vor Vandalismus oder ähnlichem steigt auch im privaten Bereich die Verbreitung von Überwachungskameras. Dabei tendiert der Markt zu immer kleineren Modellen, mit höheren Auflösungen. Geräte solcher Art gibt es bereits zu Genüge, weswegen unser Projekt sich auf einen anderen Ansatz konzentriert.\\ Ziel war es, mithilfe von zwei oder mehr Rasperry Pis, die jeweils mit Kamera und Infrarot-Sensoren bestückt sind, ein Gebiet oder Raum auf unerwünschte Eindringlinge zu überprüfen. Die Geräte sollen dazu an unterschiedlichen Stellen platziert werden, die den zu überwachenden Raum aus unterschiedlichen Blickwinkeln beobachten. Die Kamera des jeweiligen Geräts soll nun auf jegliche wahrgenommene Bewegung reagieren. Um die Genauigkeit zu erhöhen und eine Aussage über die Art der erkannten Bewegung treffen zu können, soll die Kamera außerdem mit dem Infrarot-Sensor, ("PIR-Sensor") kooperieren.\\ Der PIR-Sensor reagiert lediglich auf Objekte, die eine Wärmesignatur ausstrahlen. Wenn nun durch die Kamera und den PIR-Sensor eine Bewegung erkannt wurde, kann der Benutzer dies anhand der Handy-App nachvollziehen. Der in der App dargestellte Prozentwert würde sehr schnell steigen, schneller als bei einer Bewegung, die lediglich durch die Kamera wahrgenommen wurde. Die Wahrscheinlichkeit, dass es sich um einen Mensch oder ein Tier handelt wäre in diesem Fall sehr hoch.\\ So weit unterscheidet sich das System, abgesehen von der App für Mobiltelefone als Informationszentrum, kaum von anderen Lösungen zur Überwachung. Um eine noch höhere Genauigkeit zu erzielen, sollen die Geräte bei unserem Ansatz allerdings zusätzlich noch miteinander kommunizieren:\\ Eine einzelne, lediglich durch die Kamera des Raspberry Pi wahrgenommene Bewegung stellt das geringste Risiko dar. Zwei oder mehr unterschiedliche Raspberry Pis, welche mit Kamera und PIR-Sensor eine Bewegung feststellen, stellen das höchste Risiko dar. Wie das funktioniert und in welcher Art und Weise die Systeme miteinander kommunizieren, wird im Folgendem erklärt.




\section{System}
\subsection{Raspberry Pi}
Der Raspberry Pi, welcher in unserem Projekt die Grundlage bildet, ist ein voll funktionsfähiger PC im Scheckkartenformat. In erster Linie wurde der Raspberry Pi mit dem Ziel entwickelt, interessierten Menschen das erlernen von hardwarenaher Programmierung zu erleichtern. Jedes Gerät besitzt ein frei programmierbares General Purpose Input Output-Board ("GPIO-Board"). Das GPIO-Board stellt je nach Modell 26 oder 40 Pins zur Verfügung, von denen 17 bzw. 26 Pins frei programmierbar sind und die weiteren der Spannungsversorgung oder als Masse dienen. Die einzelnen Pins dieses Boards lassen sich mit selbst programmierten Programmen ansteuern und können vielseitigen Zwecken dienen. So kann diverse externe Peripherie angesteuert werden, wie z.B. ein Temperatur-Sensor, ein Ultraschall-Sensor, ein Motor oder sogar ein kleiner externen Monitor mit Touch-Funktion. Als Betriebssystem können unter anderem an die Architektur angepasste Linux-Distributionen wie das auf Debian basierende \textit{Raspbian} installiert werden.  Auch Betriebssysteme, welche den Raspberry Pi zum Mediencenter umfunktionieren, um damit Filme und Musik abzuspielen, sind von der Community mittlerweile zur Verfügung gestellt worden. Wie solche Projekte zeigen, ist der Raspberry Pi nicht nur zum lernen gut geeignet. Die Video- und Audioausgabe erfolgt über eine HDMI-Schnittstelle, für die Audioausgabe steht alternativ auch ein 3,5mm Klinkenanschluss zur Verfügung. Für die Stromversorgung wird ein 5-V-Micro-USB-Anschluss genutzt. Hier stehen dem Anwender viele Türen offen: \\ Neben z.B. den meisten Handy-Ladegeräten kann die Stromversorgung auch über Batterie und Solarzelle erfolgen. So kann der Raspberry Pi auch mobil verwendet werden. Bis dato konnte sich der Raspberry Pi knapp 4 Millionen mal verkaufen und ist inzwischen in seiner vierten Version erschienen. \cite{verkaufszahlen} Die erste Version dieses Rechner kam Anfang 2012 auf den Markt und erfreut sich seither größter Beliebtheit. Je nach Ausführung ist das Gerät zwischen 25 und 35 Euro teuer. Die unterschiedlichen Ausführungen unterscheiden sich in gewissen Punkten: \\ Modell A und A+ besitzen 256 MB Arbeitsspeicher und nur einen USB-Anschluss, Modell B und B+ besitzen 512 MB Arbeitsspeicher, eine Ethernet-Schnittstelle sowie zwei, respektive vier USB-Anschlüsse. Alle Modelle müssen ohne Festplattenschnittstelle auskommen und verwenden SD-Karten  bzw. Micro-SD-Karten als Speichermedium. Für unsere Ausarbeitung haben wir den Raspberry Pi B+ verwendet.\\


\subsection{Raspberry Pi Kameramodul}
Für unsere Ausarbeitung verwenden wir das Raspberry Pi Infrarot Kamera Modul ("Pi NoIR Camera Board"), weil es vom Raspberry Pi selbst auf jeden Fall unterstützt wird und der Support der Raspberry Pi Community hervorragend ist. Die Kamera bietet eine Auflösung von bis zu 5 Megapixel und kann bei statischen Aufnahmen mit einer Auflösung von bis zu 2592 x 1944 Pixel aufwarten. Mit Abmessungen von 25 x 20 x 9 mm ist die Kamera äußerst klein, muss allerdings auch ohne eigenes Gehäuse auskommen. Für die Raspberry Pi Kamera gibt es am Raspberry Pi einen eigenen Slot, in dem das Flachbandkabel der Kamera passt, sodass die GPIO-Anschlüsse am Raspberry Pi vollständig für andere Aufgaben verwendet werden können. Da die Kamera dazu in der Lage ist, Licht aus dem Infraroten Spektralbereich einzufangen, kann man auch bei schlechter Beleuchtung oder nachts noch Bewegungen erkennen lassen oder Fotos schießen. Dazu muss bei Nacht der entsprechende Bereich mit Infrarotlicht ausgeleuchtet werden. Bei vielen Überwachungskameras wird dieses Licht mittels LEDs erzeugt, welche um das Objektiv der Kamera platziert werden. Für unseren Prototypen kommen vorerst keine Infrarot-LEDs zum Einsatz, können jedoch um diese Funktion erweitert werden. 

\subsection{PIR-Sensor}
Beim zweiten Sensor handelt sich um ein passiven Infrarot Sensor:\\ \textit{Der PIR-Sensor (Passive Infrared Sensor) ist einer der gängigsten Bewegungsmelder und ist oftmals auch in bewegungssensitiven Außenleuchten oder Alarmanlagen verbaut. Erkennbar ist der PIR-Sensor an seiner meist runden, milchigen Kuppel, die mit vielen einzelnen Waben versehen ist. Der Sensor reagiert auf Temperaturveränderungen in seinem Sichtfeld. Somit können Menschen oder Tiere im Aktionsradius des Sensors erkannt werden. Jedoch kann der Sensor nur Veränderungen wahrnehmen. Bleibt ein Objekt ruhig im Bereich des Sensors stehen, so wird es nicht weiter erkannt.  Sobald es sich weiterbewegt, schlägt der Sensor erneut an.}\cite[S. 493]{raspi}\\ \\ Im inneren eines solchen Moduls befinden sich zwei Folien, die an ihrer Oberfläche unterschiedliche elektrische Ladungen aufweisen. Trifft nun die Wärmestrahlung eines bestimmten Frequenzbereichs auf diese Folien, wird deren Polarisation verschoben und eine elektronische Spannung erzeugt, welche den Sensor zum auslösen bringt. Die milchige Kuppel auf dem Sensor erweitert den Erfassungsbereich des Sensors, indem es wie eine Anordnung von Linsen fungiert und lenkt die Wärmestrahlung direkt auf eine der beiden Folien. Bewegt sich nun eine Wärmequelle durch den vom Sensor überwachten Raum, kann man eine Bewegung über einen großen Bereich und in relativ großer Entfernung registrieren.\cite{pir} Um den PIR Sensor am Raspberry Pi anzuschließen, werden insgesamt 3 GPIO Anschlüsse benötigt. Einen 3,3V Anschluss für die Stromversorgung, einen Ground und ein frei programmierbarer GPIO Pin, um den PIR-Sensor zu steuern.


\section{Codis: Kooperative, verteilte Überwachung}

Um eine kooperative, verteilte Überwachung zu realisieren, haben wir das Programm Codis entwickelt. Codis steht für cooperative, distributed surveillance\footnote{zu Deutsch: Kooperative, verteilte Überwachung}. Entdeckt Codis einen Eindringling, sendet es eine \MakeUppercase{intruder} Nachricht an alle Geräte im Netzwerk. Wie diese Nachricht verarbeitet wird und welche Maßnahmen eingeleitet werden sollen, wenn ein Eindringling entdeckt wurde, ist den Anwendern von Codis überlassen.\todo[inline]{Inwiefern kann der Anwender das?} In diesem Abschnitt befassen wir uns damit, wie Codis als verteiltes System ein Gebiet auf Eindringlinge überwacht. Dazu klären wir, wie Codis einen Eindringling entdeckt und wie aus Codis ein verteiltes System entsteht. Anschließend stellen wir fest, wie Codis als verteiltes System seine Effizienz verbessert und seine Genauigkeit beim Entdecken von Eindringlingen erhöht.

\subsection{Bewegungserkennung mit der Kamera}

Der Raspberry verwendet einen H264 Kodierer zur Videokompression. Um Videos zu kodieren, verwendet der Kodierer ein Verfahren, das als Motion Estimation\footnote{wortwörtlich: Bewegungsvorhersage}\cite{estimation} bezeichnet wird. Dabei wird das Bild in 16x16 Pixel große Quadrate eingeteilt. Diese Quadrate bezeichnet man als Makroblöcke. Beim Kodieren von Videos vergleicht der H264 Kodierer das aktuelle Bild mit einem Referenzbild. Dazu untersucht der Kodierer jeden einzelnen Makroblock im aktuellen Bild und sucht nach dem ähnlichsten Makroblock im Referenzbild. Der jeweilige Abstand zwischen den Makroblock im aktuellen Bild und im Referenzbild wird vom Kodierer gespeichert. Anhand dieses Abstands kann gemessen werden, wie sehr sich ein Makroblock im Bild gegenüber dem Referenzbild bewegt hat. Bis zu diesem Schritt wurde der komplette Vorgang im H264 Kodierer implementiert, um Bewegungen zu erkennen.\cite{vektoren} Als nächstes nehmen wir uns für jedes Bild die Bewegungsdaten als Array und berechnen damit das Ausmaß an Bewegung, die im Bild stattfindet. Dazu berechnen wir das Ausmaß der einzelnen Vektoren im Array mit dem Satz des Pythagoras'. Eine Bewegung wurde dann erkannt, wenn im Array mindestens zehn Vektoren vorhanden sind, deren Bewegungsausmaß mindestens 60 beträgt.

\subsection{Codis als verteiltes System}

Codis' Hauptfunktion ist es, ein Gebiet aus verschiedenen Blickwinkeln zu überwachen. Dieser Ansatz soll die Genauigkeit beim Erkennen von möglichen Eindringlingen erhöhen.\todo[inline]{} Dazu verwendet Codis mehrere Raspberrys, die sich in einem Netzwerk befinden und untereinander Nachrichten austauschen. Wir verwenden im Folgenden den Begriff Codis-System für die Menge aller Raspberrys, auf denen Codis läuft und die miteinander in einem Netzwerk kommunizieren. Als Koordinator bezeichnen wir ein Raspberry im Codis-System, der spezielle Aufgaben übernimmt, die wir im Laufe dieses Abschnittes klären. Der Koordinator ist allen Raspberrys im Codis-System bekannt und wird innerhalb des Codis-Systems ausgewählt. Das Codis-System wird dann erzeugt, wenn ein Raspberry Codis ausführt, während kein anderer Raspberry im Netzwerk Codis ausführt und wird dann zerstört, wenn Codis vom letzten Raspberry im Codis-System beendet wird. 

\subsubsection{Codis-Liste}

Codis verwendet eine verteilte Liste der Netzwerkadressen aller Raspberrys im Codis-System. Als verteilte Liste bezeichnen wir eine Liste, die auf allen Geräten eines verteilten Systems lokal abgespeichert ist und stets redundant zu den lokal abgespeicherten Listen der jeweils anderen Geräten ist. Die verteilte Liste, die Codis verwendet, bezeichnen wir im Folgenden als Codis-Liste. Die Codis-Liste wird zu Koordinationszwecken zwischen den Raspberrys im Codis-System benötigt.
Die Codis-Liste wird dann gebildet, wenn das Codis-System erzeugt wird. Möchte ein Raspberry dem Codis-System beitreten, sendet dieser eine \MakeUppercase{joinrequest} Nachricht an das Codis-System. Daraufhin horcht der Raspberry fünf Sekunden lang auf eine \MakeUppercase{joinresponse} Nachricht, die von allen Raspberrys im Codis-System versendet wird. Die \MakeUppercase{joinresponse} Nachricht enthält die Position des Absenders in der Codis-Liste. Nachdem der Koordinator seine \MakeUppercase{joinresponse} Nachricht versendet hat, wartet er eine Sekunde und sendet daraufhin eine \MakeUppercase{coordinator} Nachricht an den Raspberry. Die \MakeUppercase{coordinator} Nachricht, enthält die Position des Koordinators in der Codis-Liste. Erhält der Raspberry die \MakeUppercase{coordinator} Nachricht, dann trägt er ein, welche Position der Koordinator in der Codis-Liste hat. Erhält der Raspberry nach fünf Sekunden keine \MakeUppercase{joinresponse} Nachricht, trägt er sich als Erster in die Codis-Liste ein und ist somit auch der Koordinator im Codis-System. Hat der Raspberry von allen anderen Raspberrys im Codis-System eine \MakeUppercase{joinresponse} Nachricht erhalten, trägt er sich ans Ende der Codis-Liste ein und sendet eine \MakeUppercase{join} Nachricht an alle Geräte im Codis-System. Erhält ein Raspberry eine \MakeUppercase{join} Nachricht, trägt er den Absender der \MakeUppercase{join} Nachricht ans Ende seiner Codis-Liste ein.

\subsubsection{Wahl eines Koordinators}

Der Raspberry von dem aus das Codis-System erzeugt wurde, wird als erster Koordinator ausgewählt. Betreten weitere Raspberrys das Codis-System, wird ein modifizierter Ringalgorithmus\cite{verteilte1}\cite{verteilte2} ausgeführt, der alle 15 Minuten einen neuen Koordinator auswählt. Um einen logischen Ring darzustellen, verwendet Codis die Codis-Liste. Die Raspberrys im Codis-System sind anhand ihrer Position in der Codis-Liste aufsteigend, im Ring angeordnet. Wird ein neuer Koordinator ausgewählt, verschickt der derzeitige Koordinator eine \MakeUppercase{election} Nachricht, an seinem Nachfolger im Ring. Erhält ein Raspberry eine \MakeUppercase{election} Nachricht, dann trägt er sich als neuer Koordinator ein und sendet eine \MakeUppercase{coordinator} Nachricht an alle Raspberrys im Codis-System. Empfängt der Raspberry, der die \MakeUppercase{election} Nachricht gesendet hat, nach fünf Sekunden keine \MakeUppercase{coordinator} Nachricht, sendet er eine \MakeUppercase{heartbeatrequest} an seinen Nachfolger. Erhält er daraufhin eine \MakeUppercase{heartbeatresponse} Nachricht zurück, sendet er die \MakeUppercase{election} Nachricht erneut an seinen Nachfolger. Empfängt er dagegen keine \MakeUppercase{heartbeatresponse} von seinem Nachfolger, dann entfernt er diesen aus der Codis-Liste und sendet eine \MakeUppercase{listupdate} Nachricht, an das Codis-System, damit der Nachfolger aus der Codis-Liste entfernt wird. Danach versucht der Raspberry die \MakeUppercase{election} Nachricht, an seinen nächsten Nachfolger zu senden.

\subsubsection{Abwechselnde Überwachung}

Unter abwechselnder Überwachung verstehen wir, dass nur der Koordinator das Gebiet überwacht, bis der nächste Koordinator gewählt wurde. Währenddessen gelten alle anderen Raspberrys als inaktiv. Inaktive Raspberrys haben ihre Sensoren abgeschaltet und warten auf eine \MakeUppercase{election} Nachricht. Entdeckt der Koordinator einen möglichen Eindringling, dann sendet dieser eine \MakeUppercase{intruder} Nachricht an alle anderen Raspberrys und geht in einen Alarmzustand über. Empfängt ein inaktiver Raspberry diese Nachricht, geht dieser auch in einen Alarmzustand über. Im Alarmzustand sind die Sensoren des Raspberrys aktiviert. Entdeckt ein Raspberry im Alarmzustand für fünf Minuten keinen Eindringling, dann geht er wieder in den inaktiven Zustand, außer der Koordinator der den Alarmzustand lediglich verlässt. Wird ein Raspberry im Alarmzustand zum Koordinator, verbleibt er im Alarmzustand. Wird ein neuer Koordinator gewählt, verbleibt der vorherige Koordinator im Alarmzustand, falls er sich in diesem bereits befindet.

\subsection{Entdecken eines Eindringlings}

Entdeckt Codis in mehreren Bildern für einen kurzen Abstand eine Bewegung, setzt es einen speziellen Wert auf TRUE. Dieser spezielle Wert wird nach 50 Bildern, in denen keine Bewegung erkannt wurde, auf FALSE gesetzt. Ein möglicher Eindringling wird dann erkannt, wenn der PIR-Sensor Bewegungen dann erkennt, während dieser spezielle Wert auf WAHR gesetzt ist. Entdeckt ein Raspberry im Codis-System einen möglichen Eindringling, dann sendet dieser eine \MakeUppercase{intruder} Nachricht an den Koordinator. Empfängt der Koordinator eine \MakeUppercase{intruder} Nachricht, merkt er sich, von welchem Raspberry diese Nachricht versendet wurde und wann die Nachricht versendet wurde. Entdeckt der Koordinator einen möglichen Eindringling, dann prüft er, ob ein anderer Raspberry in den letzten drei Sekunden einen möglichen Eindringling entdeckt hat. Ist das der Fall, sendet der Koordinator eine \MakeUppercase{intruder} Nachricht ans Netzwerk.

\section{Andere Ansätze (bzw. Verwandte Arbeiten)}
Der Versuch, einen Raspberry Pi als kooperatives, verteiltes Überwachungssystem zu verwirklichen, ist bisher noch nicht unternommen worden und stellt insofern ein Novum dar. Es existiert jedoch eine Lösung, ein Linux-System als pure Überwachungskamera nutzbar zu machen. Das Programm Motion \cite{motion} zählt zu eben einer dieser Lösungen die wir uns für unsere Ausarbeitung angesehen haben, da es im Kern am ehesten mit unserer Ausarbeitung vergleichbar ist und die vielleicht am weitesten verbreitete Software dieser Art darstellt. \\Motion verfolgt das Ziel, aus dem Computer mit angeschlossener USB-Webcam eine Überwachungskamera zu erstellen. Wie bei Codis beruht auch hier ein Großteil der Funktionalität auf dem Erkennen von Bewegungen. Allerdings führt eine erkannte Bewegung bei Motion dazu, dass wahlweise ein Foto oder Video aufgezeichnet wird. Ebenso kann der aktuelle Video-Stream der Kamera abgegriffen werden um das aktuelle Geschehen selbst zu beobachten. Das Programm wurde nicht speziell für den Raspberry Pi geschrieben und ist dahingehend nicht für die Nutzung auf einem solchen, vergleichsweise schwachem System ausgelegt und lässt sich im Originalzustand nicht mit dem Raspberry Pi Kameramodul betreiben. Motion lässt sich durch den Anschluss mehrerer Kameras erweitern, wobei diese sich untereinander in ihrer Funktionsweise nicht beeinflussen. Codis bietet demgegenüber den Vorteil, dass das Codis-System leicht erweiterbar ist und eine höhere Fehlertoleranz bietet. Fällt ein Raspberry Pi aus, wird er aus der Codis-Liste gelöscht. Ein neuer Raspberry Pi auf dem Codis ausgeführt wird, integriert sich automatisch in das System und hilft bei der Überwachung des Gebiets.


\section{Evaluation}

Codis soll als verteiltes System nicht nur seine Genauigkeit beim Entdecken von Eindringlingen erhöhen, sondern auch die CPU-Auslastung der einzelnen Raspberrys verbessern. Dazu ist Codis' Funktion der abwechselnden Überwachung gedacht. Überwacht ein Raspberry mit Codis das Gebiet, beansprucht er ungefähr 30 Prozent an CPU Leistung, ohne einen Eindringling zu entdecken. Durch die abwechselnde Überwachung, teilen sich die Raspberrys die CPU-Last auf, indem nur einer für ein Intervall von 15 Minuten das Gebiet überwacht, bis ein Eindringling entdeckt wurde. Die Raspberrys, die mit Codis ein Gebiet im Freien überwachen, müssten mit externen Akkus betrieben werden. Durch die abwechselnde Überwachung wird der Akku eines Raspberrys weniger beansprucht.

\section{Zusammenfassung}




\printbibliography

\end{document}

