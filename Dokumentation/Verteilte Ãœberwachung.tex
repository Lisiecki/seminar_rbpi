\documentclass[9pt,a4paper]{IEEEtran}
\usepackage[utf8]{inputenc}
\usepackage{amsmath}
\usepackage{amsfonts}
\usepackage{amssymb}
\usepackage{todonotes}	%todonotes für die toto-annotationen im Text und in der Datei

%% BIBLATEX EINBINDEN 
\usepackage[backend=bibtex]{biblatex} 
\bibliography{refs}

\title{Kooperative, verteilte Überwachung eines Gebiets auf Eindringlinge}		%Beschriftung für Titelblatt
\author{Dennis Lisiecki, Torsten Kühl}								%Ebenso, nur kleiner

\begin{document}	%Hier beginnt das Dokument

\maketitle	%Titelblatt erstellen
%\tableofcontents	%Inhaltsverzeichnis erstellen....NICHT

%\chapter{Einleitung}
\todo{Raspberry erklären, GPIO erklären, }
\section{Anforderungen / Lastenheft}
Ziel ist es, mithilfe von zwei oder mehr Rasperry Pis, die jeweils mit Kamera und Infrarot-Sensoren bestückt sind, ein Gebiet oder Raum auf unerwünschte Eindringlinge zu überprüfen.\\
Dazu werden die Geräte an unterschiedlichen Positionen aufgestellt, die den entsprechenden Raum aus verschiedenen Blickwinkeln zeigen. 
Die Raspberry Pis sollen zunächst ihre Umgebung auf "Artgenossen" überprüfen und diese anhand anderer Blinkfrequenzen an ihren Infrarot-LEDs erkennen. \\
Um unerwünschte Alarme auszuschließen, soll die Kamera mit dem PIR-Sensor kooperieren. \\
Wenn nun eine Bewegung erkannt und diese ebenfalls durch das PIR-Sensor bestätigt wurde, soll ein Alarm dafür sorgen, dass eine Mitteilung an das Handy des Benutzers geschickt wird und dieser entsprechend informiert wird.\\
Ebenfalls denkbar wäre ein Mechanismus, welcher wahrgenommene Bewegungen in ihrer Dringlichkeit beurteilt:\\
Eine einzelne wahrgenommene Bewegung durch die Kamera eines einzelnen Raspberry Pi stellt das geringste Risiko dar.
Zwei unterschiedliche Raspberry Pis, welche mit Kamera und PIR-Sensor eine Bewegung feststellen, stellen das höchste Risiko dar.


\section{Bestehende Lösungen} 
Zu Beginn des Projekts haben wir als Grundlage für die Bearbeitung des Projekts das Programm Motion\cite{motion} als passend angesehen. 
Das Programm konnte nach leichter Modifikation mit der Raspberry Pi Kamera verwendet werden und besaß einen Funktionierenden Algorithmus zur Erkennung von Bewegungen. 
Außerdem stellte es einen http-Stream zur Verfügung, über welchen das aktuelle Geschehen vor der Kamera beobachtet werden und nach erkannter Bewegung automatisch Fotos und Videos aufnehmen konnte. 
\\ 
Als unüberwindlicher Nachteil stellte sich jedoch heraus, dass der Quellcode, welcher in C geschrieben ist, nicht so leicht vom Raspberry Pi neu kompiliert werden konnte, wenn z.B. Änderungen daran getätigt worden sind. 
Auch ohne Änderungen gab es Probleme bei der Kompilierung, die nach erheblichen Zeitaufwand nicht beseitigt werden konnten. 
Auch das aufzeichnen von Videomaterial hat sich als äußert Ressourcenintensiv erwiesen und bleibt deswegen für unser Projekt unbrauchbar, zumal ein Raspberry Pi nicht den nötigen Speicherplatz mitbringt, um dauerhaft aufnahmen zu machen und zu speichern.
\\ \\
Nach kurzer Recherche entschieden wir uns dafür, das Projekt in Ressourceschonendem Python-Code zu verwirklichen.
In dem Buch "Raspberry Pi - Das umfassende Handbuch" konnte ein recht minimalistischer Beispielcode für die Bewegungserkennung mit der PiNoIR-Kamera gefunden werden\cite[S. XXX]{Raspi}. \todo{Bewegungserkennung zitieren} 
Durch die öffentlich zugänglichen Projekte aus der Raspberry-Community konnten Lehren daraus gezogen werden, wie man einen möglichst effektiven Algorithmus schreibt, der unseren Ansprüchen genügt. 
Ebenfalls im Raspberry-Handbuch konnte ein Beispiel für das Arbeiten mit dem PIR-Sensor gefunden werden\cite[S. 495]{Raspi}, welchen wir in unseren Code übernehmen konnten.
Diesen wollen wir um die von uns benötigten Funktionen ergänzen.


%\chapter{Technische Realisierung}
\section{Benötigte}
%TODO Preise von Rechnungen holen ? Reich si
\begin{itemize}
\item Raspberry Pi B+ \textit{33,70 EUR} 
\item SD-Karte 8GB \textit{7 EUR}
\item PiNoIR Kamera \textit{24,58 EUR}
\item PIR-Sensor \textit{3,99 EUR}
\item Steckplatine \textit{2,33 EUR}
\item Kabel \textit{2,01 EUR}
\end{itemize}

Somit kommt man pro Einheit auf Materialkosten von ca 58 Euro.

\section{Verwendete Sensoren}
Damit ein Computer die Bewegungen in seinem Umfeld erkennen kann, benötigt er Sensoren, die in der Lage sind, die äußere Umgebung wahrzunehmen. 
Unser System verwendet zwei Sensoren, die jeweils auf eine völlig andere Art und weise "sehen". 

\subsection{Raspberry Pi Kameramodul}
Der erste Sensor ist das Raspberry Pi Kameramodul, die wir an unseren Raspberry Pi angeschlossen haben. 
Für unser Projekt verwenden wir das Raspberry Pi Infrarot Kamera Modul \textit{"Pi NoIR Camera Board"}, weil es vom Raspberry Pi selbst auf jeden Fall unterstützt wird und der Support der Raspberry Pi Community hervorragend ist. 
Natürlich können zur Bewegungserkennung auch gewöhnliche Webcams verwendet werden. 
Für die Raspberry Pi Kamera gibt es am Raspberry Pi einen eigenen Slot, in dem das Flachbandkabel der Kamera passt, sodass die GPIO-Anschlüsse am Raspberry Pi vollständig für andere Aufgaben verwendet werden können.
Wir geben zu bedenken, dass das Python Script, das wir zur Bewegungserkennung verwenden, nur das Raspberry Pi Kamera Modul unterstützt. 
Da die Kamera dazu in der Lage ist, Licht aus dem Infraroten Spektralbereich zu sehen, kann man auch nach Sonnenuntergang noch Bewegungen erkennen.
Dazu muss der entsprechende Bereich mit Infrarotlicht ausgeleuchtet werden.
Bei vielen Überwachungskameras wird dieses Licht mittels LEDs, welche um das Objektiv der Kamera platziert werden, erzeugt.

\subsection{PIR-Sensor}
Beim zweiten Sensor handelt sich um ein pyroelektrischen Infrarot Sensor:\\
\textit{Der PIR-Sensor (Passive Infrared Sensor) ist einer der gängigsten Bewegungsmelder und ist oftmals auch in bewegungssensitiven Außenleuchten oder Alarmanlagen verbaut.
Erkennbar ist der PIR-Sensor an seiner meist runden, milchigen Kuppel, die mit vielen einzelnen Waben versehen ist. 
Der Sensor reagiert auf Temperaturveränderungen in seinem Sichtfeld. 
Somit können Menschen oder Tiere im Aktionsradius des Sensors erkannt werden. 
Jedoch kann der Sensor nur Veränderungen wahrnehmen. 
Bleibt ein Objekt ruhig im Bereich des Sensors stehen, so wird es nicht weiter erkannt. 
Sobald es sich weiterbewegt, schlägt der Sensor erneut an.}\cite[S. 493]{Raspi}\\ \\
\\
Laut Verkäufer sind die PIR Sensoren, die wir verwenden, für den menschlichen Körper gedacht. \todo[inline]{Diesen Teil vllt damit ersetzen, dass wir den PIR erfolgreich an Tieren getestet haben?}
Allerdings beträgt die Abweichung der Körpertemperaturen von Säugetieren nur wenige Grad Celsius.
Dementsprechend sollte der PIR auch in der Lage sein, z. B. einen Luchs zu erkennen.
Um den PIR Sensor am Raspberry Pi anzuschließen, werden insgesamt 3 GPIO Anschlüssen benötigt.
Einmal ein 3V Anschluss für die Stromversorgung, einen Ground und ein regulärer GPIO Pin, um den PIR zu steuern.

%\chapter{Theoretische Herangehensweise}
\section{Funktionsweise der Hard- und Software}
\todo[inline]{Funktionsweise noch nicht final}
Wenn das Programm ausgeführt wird, soll der Raspberry zunächst seine Umgebung auf andere anwesende Raspberries überprüfen und deren Koordinaten mithilfe des Magnetometer speichern. 
Anschließend soll die Kamera auf Bewegung achten. 
Sobald die Kamera eine Bewegung erkannt hat, soll das PIR-Sensor prüfen, ob in der nähe auch eine wahrnehmbare Bewegung im Infraroten Bereich erfolgt ist. 
Ist auch dies der Fall, reden wir von einer definitiven Bewegung. 
Die Himmelsrichtung wird an die anderen anwesenden Systeme geschickt, welche dann (falls nötig) ebenfalls in diese Richtung schwenken. 
Am Raspberry geht dann noch eine LED an, um eine erkannte Bewegung zu signalisieren. 

\section{Algorithmus für die Bewegungserkennung}


\section{Überlegungen für ein verteiltes System}


%\chapter{Praktische Realisierung}

\section{Das Erkennen von Bewegungen}


\section{Überwachen eines Gebiets als verteiltes System}


\printbibliography

\end{document}